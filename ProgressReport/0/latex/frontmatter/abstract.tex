
\cleardoublepage
\phantomsection
\begin{center}
	\section*{Abstract}
	\addcontentsline{toc}{section}{\numberline{}Abstract}%
\end{center}

The recent demand for commercial sub-orbital and orbital flight has increased the prevalence of supersonic propulsion. A key component of this in rocket engines is the converging-diverging nozzle which accelerates combustion gases to supersonic speeds to deliver the thrust required during lift-off and flight. The shape of this nozzle influences the flow conditions and must be properly designed to provide the maximum thrust in a nozzle body with the lowest weight. 

In this thesis, flow in converging-diverging nozzles has been taken into account. One dimensional isentropic flow is analyzed by developing a custom programming library. For two dimensional analysis, the method of characteristics is utilized to obtain shock-free, minimum length wall contour for the diverging section of a supersonic nozzle. This optimal wall contour has been simulated in OpenFOAM finite volume code and a compressible flow solver. Comparative analysis is showed good agreement between method of characteristics and OpenFOAM simulations considering Mach number and temperature distributions of the diverging part. Proposed approaches, developed and implemented using open-source tools, provide a trustworthy solution for predicting the optimal nozzle design and internal flow behaviour under supersonic conditions. 

\afterpage{\blankpage}

