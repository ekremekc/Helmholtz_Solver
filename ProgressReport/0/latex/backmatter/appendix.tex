\cleardoublepage
\phantomsection
\addcontentsline{toc}{section}{Appendix}
\begin{appendices}

\section{Nomenclature}

\noindent $Ma$ : Mach number

\noindent $\alpha$ : speed of sound ($\frac{m}{s}$)

\noindent $k$ : specific heat ratio

\noindent $R$ : specific gas constant ($\frac{J}{kg.K}$)

\noindent $T$ : temperature (K)

\noindent $T_0$ : stagnation temperature (K) 

\noindent $P$ : pressure (Pa)

\noindent $P_0$ : stagnation pressure (Pa) 

\noindent $\rho$ : density ($\frac{kg}{m^3}$)

\noindent $\rho_0$ : stagnation density ($\frac{kg}{m^3}$) 

\noindent $V$ : velocity ($\frac{m}{s}$)

\noindent $A$ : area ($m^2$)

\noindent $A^*$ : critical throat area ($m^2$)

\noindent $\phi$ : velocity potential

\noindent $\theta$ : deflection angle

\noindent $\nu$ : Prandtl-Meyer angle

\noindent $\mu$ : Mach angle

\noindent $\Psi$ : any scalar property
\afterpage{\blankpage}
\newpage

\renewcommand{\thefigure}{A.\arabic{figure}}
\setcounter{figure}{0}
\section{Method of Characteristics Procedure}

In this appendix, method of characteristics procedure is explained step by step by considered supersonic nozzle in this study. Expansion waves are assumed as symmetric according to x axis, then calculations has been made for upper half of the nozzle.

\FloatBarrier
\begin{figure}[!htb]
	\centering
	\includegraphics[scale=0.30]{a1.png}
	\caption{Mach number profile}	

\end{figure}
\FloatBarrier

\noindent {\LARGE \textbf{Tutorial Case}}\\

\noindent Exit Mach number is considered as 2.4 and number of characteristics has taken as N=7. Numbered intersection points, wall points and axis points can be seen in Figure \ref{fig:a2}.

\FloatBarrier
\begin{figure}[!htb]
	\centering
	\includegraphics[scale=0.35]{a2.png}
	\caption{Numbered nozzle profile}	
	\label{fig:a2}
\end{figure}
\FloatBarrier

\pagebreak

\FloatBarrier
\begin{figure}[!htb]
	\centering
	\includegraphics[scale=0.4]{a3.png}
	\caption{Nozzle profile}	
	\label{fig:a3}
\end{figure}
\FloatBarrier

Prandtl-Meyer(PM) angle at the exit for Mach number 2.4;

 $\nu_e = 36.746$

$cd$ characteristic line has same stream properties; 

$\nu_e = \nu_c = 36.746$

Downstream of $cd$ characteristic line is parallel to the x-axis; 

$\theta_e = \theta_c = 0$

Right running characteristic on point $c$ ; 

$K_c^- = \theta_c + \nu_e = 36.746$

Point $c$ and point $a$ on same characteristic $ac$;

$K_c^- = K_a^- = 36.746$

Occurred PM expansion waves of corner $a$;

$\nu_a = \nu(1)+\Delta\theta$

where $\nu(1)$ is downstream of PM waves, $\nu_a$ is upstream of PM waves and 

$\Delta\theta$ is deviation of first wave.\\

\begin{minipage}{0.5\textwidth}
	
	\textbf{For corner $a$};\\
	
	Mach number;
	
	$M = 1$
	
	PM of this Mach number;
	
	$\nu(1) = 0$
	
	Deviation of first wave;
	
	$\Delta\theta = \theta_{max}$
	
	Downstream PM angle;
	
	$\nu_a = \nu(1) + \theta_{max}$
	
	$\nu_a = \theta_{max}$
	
	On characteristic line $ac$, point $a$;
	
	$\theta_a + \nu_a = K_a^-$
	
	$\theta_a = \theta_{max}$ and $\nu_a = \theta_{max}$, then;
	
	$\theta_{max} + \theta_{max} = K_a^-$
	
	Therefore; 
	
	$\theta_{max} = \frac{\nu_e}{2}=\frac{36.746}{2}=18.373$
	
\end{minipage}
\begin{minipage}{0.5\textwidth}
	\includegraphics[width=\linewidth]{a4.PNG}
	\captionof{figure}{First wave}
\end{minipage}%
\noindent

\pagebreak

\noindent First expansion wave which propagates from point $a$ is selected by doing small angle of $\Delta\theta$. On this line, deflection angles are same from point $a$ to point $1$. This assumption causes tolerable error of method of characteristics technique, needs to be as small as possible. To illustrate, first deviation angle is selected as  $\Delta\theta = 0.373$. Remaining angle $18.373 - 0.373 = 18$ can be divided into equally spaced 6 parts.  

\FloatBarrier
\begin{figure}[!htb]
	\centering
	\includegraphics[scale=0.6]{a5.png}
	\caption{Characteristic of first wave}	
	\label{fig:a5}
\end{figure}
\FloatBarrier

\noindent\textbf{Determination of Point 1};\\

\begin{minipage}{0.65\textwidth}
	
	
	For point $a$ $(M=1, \Delta\theta = 0.373)$;
	
	$\nu_a = \nu(1)+\Delta\theta = 0.373$
	
	$M_a = 1.0417$
	
	$\mu_a = 73.736$
	
	On right running characteristic line of point $a$;
	
	$K_a^- = \theta_a+\nu_a = \Delta\theta + \nu_a = 0.373+0.373=0.746$
	
	On left running characteristic line of point $b$;
	
	$K_b^+ = \theta_b-\nu_b = -\Delta\theta - \nu_b = -0.373-0.373=-0.746$
	
	On right running characteristic line of $a-1$;
	
	$K_1^- = K_a^- = 0.746$
	
	On left running characteristic line of $b-1$;
	
	$K_1^+ = K_b^+ = -0.746$
		
\end{minipage}
\begin{minipage}{0.35\textwidth}
	\includegraphics[scale=0.4]{a6.PNG}
	\captionof{figure}{First wave}
\end{minipage}%
\noindent

\pagebreak

Therefore, point $1$ becomes;\\

$\theta_1 = \frac{K_1^-+K_1^+}{2}=\frac{0.746-0.746}{2}=0$

$\nu_1=\frac{K_1^--K_1^+}{2}=\frac{0.746+0.746}{2}=0.746$

From equation \eqref{eqn:pm_angle}, $M_1=1.0669$

From equation \eqref{eqn:mu}, $\mu_1=69.605$\\

Slope of $a-1$ characteristic;

${\alpha}^{-}_{\alpha-1} = \frac{\theta_a+\theta_1}{2}-\frac{\mu_a+\mu_1}{2} = \frac{0.373+0}{2}-\frac{73.736+69.605}{2}$

${\alpha}_{\alpha-1}^{-} = -71.484 $ $\rightarrow$  $m_{\alpha-1} = tan(-71.484) = -2.986$\\

Radius of throat is assumed as 1;\\

{\color{magenta}
$x_1 = \frac{-1}{-2.986} = 0.3349 $

$y_1 = 0$
}\\

\noindent\textbf{Determination of Point 2};\\

\begin{minipage}{0.65\textwidth}
	
$M=1$ and $\Delta\theta = 3.373$ gives $\nu(M_2)=3.373$. Then;

$M_2=1.1924$

$\mu(M_2)=56.995$

Along $a-2$ characteristic line;

$K_2^- = \theta + \nu = 3.373 + 3.373 = 6.746$

Along $C^+$ characteristic line;

$K_2^+ = K_1^+ = -0.746$

According to this, at point $2$;

$\theta_2 = \frac{K_2^-+K_2^+}{2} = \frac{6.746-0.746}{2} = 3$

$\nu_2 = \frac{K_2^--K_2^+}{2} = \frac{6.746+0.746}{2} = 3.746$

$M_2 = 1.208$

$\mu_2 = 55.903$\\
	
\end{minipage}
\begin{minipage}{0.35\textwidth}
	\includegraphics[scale=0.4]{a7.PNG}
	\captionof{figure}{Reflection for point 2}
\end{minipage}%
\noindent


Slope of $a-2$ characteristic;

${\alpha}^{-}_{\alpha-2} = \frac{\theta_a+\theta_2}{2}-\frac{\mu_a+\mu_2}{2} = \frac{3.373+3}{2}-\frac{56.995+55.903}{2}$

${\alpha}_{\alpha-2}^{-} = -53.263 $ $\rightarrow$   $m_{\alpha-2} = tan(-53.263) = -1.3398$\\

Slope of $1-2$ characteristic;

${\alpha}^{+}_{1-2} = \frac{\theta_1+\theta_2}{2}+\frac{\mu_1+\mu_2}{2} = \frac{0+3}{2}+\frac{69.605+55.903}{2}$

${\alpha}_{1-2}^{+} = 64.254 $  $\rightarrow$   $m_{1-2} = tan(64.254) = 2.0736$\\

\pagebreak

\begin{minipage}{0.65\textwidth}
	
	Equation of $a-2$ line;
	
	$y=y_a+m_{s-2}(x-x_a)$
	
	Equation of $1-2$ line;
	
	$y=y_1+m_{1-2}(x-x_1)$
	
	At intersection point $2$;
	
	$y_2 = y_a + m_{a-2}(x_2-x_a)=y_1+m_{1-2}(x_2-x_1)$
	
	Then $x_2$ becomes;\\
	
	$x_2=\frac{y_a-y_1+m_{1-2}x_1-m_{a-2}x_a}{m_{1-2}-m_{a-2}}$\\
	
	According to these equations;\\
	
	{\color{magenta}
		$x_2 = \frac{1-0+2.0736(0.3349)-0}{2.0736-(-1.3398)} = 0.4964 $
		
		$y_2 = 0+2.0736(0.4964-0.3349)=0.3349$
	}\\
	
\end{minipage}
\begin{minipage}{0.35\textwidth}
	\includegraphics[scale=0.4]{a8.PNG}
	\captionof{figure}{Calculations for point 2}
\end{minipage}%
\noindent

\noindent\textbf{Determination of Point 3};\\

\begin{minipage}{0.65\textwidth}
	
	$M=1$ and $\Delta\theta = 6.373$ gives $M_2=1.3074$. Then;
	
	$\nu(M_2)=6.373$
	
	$\mu(M_2)=49.896$
	
	Along $a-3$ characteristic line;
	
	$K_3^- = \theta + \nu = 6.373 + 6.373 = 12.746$
	
	Along $C^+$ characteristic line;
	
	$K_3^+ = K_2^+ = -0.746$
	
	According to this, at point $3$;
	
	$\theta_3 = \frac{K_3^-+K_3^+}{2} = \frac{12.746-0.746}{2} = 6$
	
	$\nu_3 = \frac{K_3^--K_3^+}{2} = \frac{12.746+0.746}{2} = 6.746$
	
	$M_3 = 1.321$
	
	$\mu_3 = 49.206$\\
	
\end{minipage}
\begin{minipage}{0.35\textwidth}
	\includegraphics[scale=0.4]{a9.PNG}
	\captionof{figure}{Calculations for point 3}
\end{minipage}%
\noindent

Slope of $a-3$ characteristic;

${\alpha}^{-}_{\alpha-3} = \frac{\theta_a+\theta_3}{2}-\frac{\mu_a+\mu_3}{2} = \frac{6.373+6}{2}-\frac{49.896+59.206}{2}$

${\alpha}_{\alpha-3}^{-} = -43.364 $ $\rightarrow$   $m_{\alpha-3} = tan(-43.364) = -0.9445$

Slope of $2-3$ characteristic;

${\alpha}^{+}_{2-3} = \frac{\theta_2+\theta_3}{2}+\frac{\mu_2+\mu_3}{2} = \frac{3+6}{2}+\frac{55.903+49.206}{2}$

${\alpha}_{2-3}^{+} = 57.055 $  $\rightarrow$  $m_{2-3} = tan(57.055) = 1.5431$\\

Coordinates of point $3$ becomes;\\

{\color{magenta}
	$x_3 = \frac{1-0.3349+1.5431(0.4964)-0}{1.5431-(-0.9445)} = 0.5753 $
	
	$y_3 = 0.3349+1.5431(0.5753-0.4964)=0.4566$
}

\pagebreak

\noindent\textbf{Determination of Point 4};\\

\begin{minipage}{0.65\textwidth}
	
	$M=1$ and $\Delta\theta = 9.373$ gives $M_3=1.4134$. Then;
	
	$\nu(M_3)=9.373$
	
	$\mu(M_3)=45.034$
	
	Along $a-4$ characteristic line;
	
	$K_4^- = \theta + \nu = 9.373 + 9.373 = 18.746$
	
	Along $C^+$ characteristic line;
	
	$K_4^+ = K_3^+ = -0.746$
	
	According to this, at point $4$;
	
	$\theta_4 = \frac{K_4^-+K_4^+}{2} = \frac{18.746-0.746}{2} = 9$
	
	$\nu_4 = \frac{K_4^--K_4^+}{2} = \frac{18.746+0.746}{2} = 9.746$
	
	$M_4 = 1.426$
	
	$\mu_4 = 44.519$\\
	
\end{minipage}
\begin{minipage}{0.35\textwidth}
	\includegraphics[scale=0.4]{a9.PNG}
	\captionof{figure}{Calculations for point 4}
\end{minipage}%
\noindent

Slope of $a-4$ characteristic;

${\alpha}^{-}_{\alpha-4} = \frac{\theta_a+\theta_4}{2}-\frac{\mu_a+\mu_4}{2} = \frac{9.373+9}{2}-\frac{45.034+44.519}{2}$

${\alpha}_{\alpha-4}^{-} = -35.590 $ $\rightarrow$  $m_{\alpha-4} = tan(-35.590) = -0.7157$

Slope of $3-4$ characteristic;

${\alpha}^{+}_{3-4} = \frac{\theta_3+\theta_4}{2}+\frac{\mu_3+\mu_4}{2} = \frac{6+9}{2}+\frac{49.206+44.519}{2}$

${\alpha}_{3-4}^{+} = 54.362 $ $\rightarrow$  $m_{3-4} = tan(54.362) = 1.3948$\\

Coordinates of point $4$ becomes;\\

{\color{magenta}
	$x_4 = \frac{1-0.4566+1.3948(0.5753)-0}{1.3948-(-0.7157)} = 0.6377 $
	
	$y_4 = 0.4566+1.3948(0.6377-0.5753)=0.5436$
}\\

\noindent\textbf{Determination of Point 5};\\

$M=1$ and $\Delta\theta = 12.373$ gives $M_4=1.5159$. Then;

$\nu(M_4)=12.373$

$\mu(M_4)=41.276$

Along $a-5$ characteristic line;

$K_5^- = \theta + \nu = 12.373 + 12.373 = 24.746$

Along $C^+$ characteristic line;

$K_5^+ = K_4^+ = -0.746$

According to this, at point $5$;

$\theta_5 = \frac{K_5^-+K_5^+}{2} = \frac{24.746-0.746}{2} = 12$

$\nu_5 = \frac{K_5^--K_5^+}{2} = \frac{24.746+0.746}{2} = 12.746$

$M_5 = 1.529$	$\rightarrow$	$\mu_5 = 40.861$

\pagebreak

\begin{minipage}{0.65\textwidth}
	
Slope of $a-5$ characteristic;

${\alpha}^{-}_{\alpha-5} = \frac{\theta_a+\theta_5}{2}-\frac{\mu_a+\mu_5}{2} = \frac{12.373+12}{2}-\frac{41.276+40.861}{2}$

${\alpha}_{\alpha-5}^{-} = -28.882 $   

$m_{\alpha-5} = tan(-28.882) = -0.5516$

Slope of $4-5$ characteristic;

${\alpha}^{+}_{4-5} = \frac{\theta_4+\theta_5}{2}+\frac{\mu_4+\mu_5}{2} = \frac{9+12}{2}+\frac{44.519+40.861}{2}$

${\alpha}_{4-5}^{+} = 53.190 $ $\rightarrow$ $m_{4-5} = tan(53.190) = 1.3363$\\

Coordinates of point $5$ becomes;\\

{\color{magenta}
	$x_5 = \frac{1-0.5436+1.33363(0.6377)-0}{1.3363-(-0.5516)} = 0.6931 $
	
	$y_5 = 0.5436+1.3363(0.6931-0.6377)=0.6177$
}\\
	
\end{minipage}
\begin{minipage}{0.35\textwidth}
	\includegraphics[scale=0.4]{a11.PNG}
	\captionof{figure}{Calculations for point 5}
\end{minipage}%
\noindent

\noindent\textbf{Determination of Point 6};\\

\begin{minipage}{0.65\textwidth}
	
	$M=1$ and $\Delta\theta = 15.373$ gives $M_5=1.6173$. Then;
	
	$\nu(M_5)=15.373$
	
	$\mu(M_5)=38.192$
	
	Along $a-6$ characteristic line;
	
	$K_6^- = \theta + \nu = 15.373 + 15.373 = 30.746$
	
	Along $C^+$ characteristic line;
	
	$K_6^+ = K_5^+ = -0.746$\\
	
	According to this, at point $6$;
	
	$\theta_6 = \frac{K_6^-+K_6^+}{2} = \frac{30.746-0.746}{2} = 15$
	
	$\nu_6 = \frac{K_6^--K_6^+}{2} = \frac{30.746+0.746}{2} = 15.746$
	
	$M_6 = 1.630$
	
	$\mu_6 = 37.844$\\
	
\end{minipage}
\begin{minipage}{0.35\textwidth}
	\includegraphics[scale=0.4]{a12.PNG}
	\captionof{figure}{Calculations for point 6}
\end{minipage}%
\noindent

Slope of $a-6$ characteristic;

${\alpha}^{-}_{\alpha-6} = \frac{\theta_a+\theta_6}{2}-\frac{\mu_a+\mu_6}{2} = \frac{15.373+15}{2}-\frac{38.192+27.844}{2}$

${\alpha}_{\alpha-6}^{-} = -22.831 $ $\rightarrow$   $m_{\alpha-6} = tan(-22.831) = -0.4210$

Slope of $5-6$ characteristic;

${\alpha}^{+}_{5-6} = \frac{\theta_5+\theta_6}{2}+\frac{\mu_5+\mu_6}{2} = \frac{12+15}{2}+\frac{40.861+37.844}{2}$

${\alpha}_{5-6}^{+} = 52.853 $ $\rightarrow$   $m_{5-6} = tan(52.853) = 1.320$\\

Coordinates of point $6$ becomes;\\

{\color{magenta}
	$x_6 = \frac{1-0.6177+1.320(0.6931)-0}{1.320-(-0.421)} = 0.7451 $
	
	$y_6 = 0.6177+1.320(0.7451-0.6931)=0.6863$
}

\pagebreak

\noindent\textbf{Determination of Point 7};\\

\begin{minipage}{0.65\textwidth}
	
	$M=1$ and $\Delta\theta = 18.373$ gives $M_6=1.7192$. Then;
	
	$\nu(M_6)=18.373$
	
	$\mu(M_6)=35.568$
	
	Along $a-7$ characteristic line;
	
	$K_7^- = \theta + \nu = 18.373 + 18.373 = 36.746$
	
	Along $C^+$ characteristic line;
	
	$K_7^+ = K_6^+ = -0.746$\\
	
	According to this, at point $7$;
	
	$\theta_7 = \frac{K_7^-+K_7^+}{2} = \frac{36.746-0.746}{2} = 18$
	
	$\nu_7 = \frac{K_7^--K_7^+}{2} = \frac{36.746+0.746}{2} = 18.746$
	
	$M_7 = 1.732$
	
	$\mu_7 = 35.267$\\
	
\end{minipage}
\begin{minipage}{0.35\textwidth}
	\includegraphics[scale=0.4]{a13.PNG}
	\captionof{figure}{Calculations for point 7}
\end{minipage}%
\noindent

Slope of $a-7$ characteristic;

${\alpha}^{-}_{\alpha-7} = \frac{\theta_a+\theta_7}{2}-\frac{\mu_a+\mu_7}{2} = \frac{18.373+18}{2}-\frac{35.568+35.267}{2}$

${\alpha}_{\alpha-7}^{-} = -17.230 $,   $m_{\alpha-7} = tan(-17.230) = -0.3101$

Slope of $6-7$ characteristic;

${\alpha}^{+}_{6-7} = \frac{\theta_6+\theta_7}{2}+\frac{\mu_6+\mu_7}{2} = \frac{15+18}{2}+\frac{37.844+35.267}{2}$

${\alpha}_{6-7}^{+} = 53.055 $ $\rightarrow$   $m_{6-7} = tan(53.055) = 1.330$\\

Coordinates of point $7$ becomes;\\

{\color{magenta}
	$x_7 = \frac{1-0.6863+1.330(0.7451)-0}{1.330-(-0.3101)} = 0.7955 $
	
	$y_7 = 0.6863+1.3297(0.7955-0.7451)=0.7533$
}\\

\noindent\textbf{Determination of Point 8 on the wall};\\

\noindent Wall between points $a$ and $8$ is assumed as linear. Then slope of the line is calculated by averaging slopes of these points. At point $a$, slope is equal to $\theta_{max}$ (Next wall point's slope is $\theta_{i}-\frac{\theta_{max}}{N}$). Between points $7$ and $8$, stream properties are same along the line, so slope of points $7$ and $8$ becomes equal. According to these;

\begin{multicols}{2}
	Slope of $a-8$ characteristic;
	
	${\alpha}^{-}_{\alpha-8} = \frac{\theta_{max}+\theta_7}{2} =18.187 $
	
	$m_{\alpha-8} = tan(18.187) = 0.3285$
	
	\columnbreak
	
	Slope of $7-8$ characteristic;
	
	${\alpha}^{+}_{7-8} = \frac{\theta_7+\theta_8}{2}+\frac{\mu_7+\mu_8}{2} = \theta_7+\mu_7$
	
	$\theta_7+\mu_7=53.267$
	
	$m_{7-8} = tan(53.267) = 1.340$
\end{multicols}

\pagebreak

Coordinates of point $8$ becomes;\\

{\color{magenta}
	$x_8 = \frac{1-0.7533+1.340(0.7955)-0.329\times0}{1.340-(0.329)} = 1.2977 $
	
	$y_8 = 0.7533-1.340(1.2977-0.7955)=1.4263$
}\\

\noindent\textbf{Determination of Point 9};\\

\begin{minipage}{0.65\textwidth}
	
	Point $9$ can be calculated via point $2$ and point $1$;\\
	
	On right running characteristic line of $2-9$;
	
	$K_9^+ = K_7^+ = 6.746$
	
	On left running characteristic line of $2^{'}-9$;
	
	$K_9^- = K_{2^{'}}^- = -6.746$\\
	
	Therefore, point $9$ becomes;\\
	
	$\theta_9 = \frac{K_9^-+K_9^+}{2}=\frac{6.746-6.746}{2}=0$
	
	$\nu_9=\frac{K_9^--K_9^+}{2}=\frac{6.746+6.746}{2}=6.746$
	
	$\nu_9 = 6.746$
	
	$M_9=1.3209$
	
	$\mu_9=49.206$
	
	Slope of $2-9$ characteristic;\\
	
\end{minipage}
\begin{minipage}{0.35\textwidth}
	\includegraphics[scale=0.4]{a14.PNG}
	\captionof{figure}{Calculation of point 9}
\end{minipage}%
\noindent



${\alpha}^{-}_{2-9} = \frac{\theta_2+\theta_9}{2}-\frac{\mu_2+\mu_9}{2} = \frac{3+0}{2}-\frac{55.903+49.206}{2}$

${\alpha}_{2-9}^{-} = -51.054 $ $\rightarrow$  $m_{2-9} = tan(-51.054) = -1.2373$\\

Coordinates of point $9$ becomes;\\


	{\color{magenta}
		$x_9 = x_2-\frac{y_2}{m_{2-9}} = 0.4964 - \frac{0.3349}{-1.2373}=0.7671 $\\
		
		$y_9 = y_1 = 0$
	}\\

\noindent\rule{14cm}{0.8pt}\\

$\star$ Calculations can be repeated for remaining undetermined points to determine entire supersonic nozzle wall contour. 

\newpage
\section{Developed Computer Algorithms }
Developed one dimensional compressible flow library can be accessed via author's GitHub profile using following link;\\

\url{https://github.com/ekremekc/compressibleflow}\\

\noindent Programmed method of characteristics technique can also be reached by author's MoC repository on GitHub.\\

\url{https://github.com/ekremekc/MoC}

\end{appendices}
\afterpage{\blankpage}

